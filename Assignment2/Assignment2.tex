\documentclass{article}
\usepackage[utf8]{inputenc}

\title{Ass2}
\author{Venkata Sai Rajesh Malladi}
\date{FPGA Assignment-2}

\begin{document}

\maketitle

\section{Introduction}

We have to perform the problem presented in Assignment-1 on arduino and verify the output.

\section{Code}

\begin{verbatim}

#include <Arduino.h>

unsigned char num ;                             //input number
unsigned char _A=0x00,_B=0x00,_C=0x00,_D=0x00;      //binary inputs
unsigned char one = 0x01;
unsigned char Org, Nor;                         //outputs
char buffer[40];
unsigned char a,b,c,d,e,f,g;

void disp_num(unsigned char A, unsigned char B, unsigned char C,unsigned char D){

a = ~( ((A)&(~B)&(~C)) | ((~A)&(B)&(D)) | ((A)&(~D)) | ((~A)&(C)) | ((B)&(C)) 
| ((~B)&(~D)) );
  b = ~( ((~A)&(~C)&(~D)) | ((~A)&(C)&(D)) | ((A)&(~C)&(D)) | ((~B)&(~C)) | 
  
  ((~B)&(~D)) );
  c = ~( ((~B)&(~C)) | ((~A)&(D)) | ((D)&(~C)) | ((~A)&(B)) | ((A)&(~B)) );
  d = ~( ((~A)&(~B)&(~D)) | ((~B)&(C)&(D)) | ((B)&(~C)&(D)) | ((B)&(C)&(~D)) | 
  
  ((A)&(~C)) );
  
  e = ~( ((~D)&(~B)) | ((C)&(~D)) | ((A)&(C)) | ((A)&(B)) );
  f = ~( ((~A)&(B)) | ((~C)&(~D)) | ((B)&(~D)) | ((A)&(~B)) | ((A)&(C)) );
  g = ~( ((~A)&(B)&(~C)) | ((C)&(~B)) | ((C)&(~D)) | ((A)&(~B)) | ((A)&(D)) );

  digitalWrite(2,one&a);
  digitalWrite(3,one&b);
  digitalWrite(4,one&c);
  digitalWrite(5,one&d);
  digitalWrite(6,one&e);
  digitalWrite(7,one&f);
  digitalWrite(8,one&g);

}
void setup() {
  pinMode(2,OUTPUT);    //a
  pinMode(3,OUTPUT);    //b
  pinMode(4,OUTPUT);    //c
  pinMode(5,OUTPUT);    //d
  pinMode(6,OUTPUT);    //e
  pinMode(7,OUTPUT);    //f
  pinMode(8,OUTPUT);    //g
  pinMode(11,OUTPUT);   //org output
  pinMode(12,OUTPUT);   //Nor output
  Serial.begin(9600);

}

unsigned char NAND(unsigned char X, unsigned char Y){ return ~(X&Y); } //NOR function

void loop() {

  Serial.println("\n\nBinary Input  (A+B)(C+D)  NOR Equivalent");

  for (num = 0x00; num<0x08; num++){
        //loop to iterate through all usecases
        delay(1000);

        _B = num>>2;    _C = num>>1;    _D = num>>0;     //changing the inputs , D is LSB
        disp_num(0, _B, _C,_D);
        Org = (_B&_C)|(_D&_C)  ;                        //Original Boolean Function
        Nor = NAND( NAND(_B, _C), NAND(_C, _D) );          //NOR gate equivalent Boolean Function
	   digitalWrite(11,one&Org);
        digitalWrite(12,one&Nor);

        sprintf(buffer, "  %x %x %x", one&_A, one&_B, one&_C);
        Serial.print(buffer);    //Input ABCD
        sprintf(buffer, "\t%x\t\t%x" , one&Org, one&Nor);
        Serial.println(buffer);          //Output Org, Nor

    }

\end{verbatim}

\section{Result}

\begin{verbatim}
The assignment has been completed and verified.
\end{verbatim}

\end{document}
